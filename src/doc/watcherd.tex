\newpage
\label{watcherd}
\section{The Watcher Daemon}

The Watcher Daemon is responsible for collecting events from the Test Nodes and
sending event streams to the Watcher GUIs.  Events from the Test Nodes are
stored in an SQLite databased (named "event.db" by default).  The Watcher Daemon
determines whether a connection is a Test Node or GUI by the type of the first
event received.

When recording events from the Test Nodes into the database, events are
appended to the existing database, or a new database is created if it does not
exist.  The Watcher Daemon may also be invoked in read-only database mode using
the command line option {\tt -r} or {\tt --read-only}, in which case events are not stored
in the database.  Read-only mode is useful particularly when replaying events
from a database from some time in the past.  In this case, it may not make
sense to append any current event stream from the Test Nodes when a large time
gap exists between past and present runs.

\subsection{Live Mode}

When a GUI connects to the Watcher Daemon, it will by default subscribe to the
live event stream coming from the Test Nodes.  In this case, the Watcher Daemon
is simply retransmitting received events to all listening GUIs rather than
fetching events from the database.  The events are also stored in the database
for later replay.

If a GUI pauses, rewinds, or slows playback, it will switch to Playback mode.

\subsection{Playback Mode}

In Playback Mode, the Watcher Daemon fetches events from the database and sends
them to the GUI.  Each GUI connection has an independent notion of the current
playback time offset, direction and speed.  Stopping playback in one GUI will
not cause playback to stop in another GUI.

The Watcher Daemon will automatically pause Playback when the last event from
the database has been sent to the GUI.  Thus, if a GUI were playing at a time
offset near the end of the database, and faster than real time, the GUI will be
paused when the last event is sent, even if additional events arrive from Test
Nodes.

\subsection{Configuration}

The Watcher Daemon has several optional configuration options.  By default,
watcherd will read its configuration from a file named {\tt watcherd.cfg}.  An
alternate configuration file can be specified using the {\tt -c} command line
option.

\begin{itemize}
\item {\tt server}, the DNS name or IP address of the network interface to listen for
connections
\item {\tt port}, the TCP port number to listen on for connections (default: 8095)
\item {\tt serverThreadNum}, the number of threads to spawn for handling connections
\item {\tt databasePath}, the full pathname to the test node event database (default:
{\tt event.db})
\end{itemize}

\subsection{Command Line Options}

\begin{itemize}
\item {\tt -c} or {\tt --config}, specify an alternate configuration file (default: {\tt
watcherd.cfg})
\item {\tt -r} or {\tt --read-only}, mark the event database Read Only so that no
new events from Test Nodes will be written
\item {\tt -h} or {\tt --help}, display a list of all supported command line
options
\end{itemize}

