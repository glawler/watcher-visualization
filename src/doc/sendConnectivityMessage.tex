\newpage
\label{sendConnectivityMessage}
\subsubsection{sendConnectivityMessage}
{\bf sendConnectivityMessage} is a test node command line program that sends a watcher::event::ConnectivityMessage message to the watcher daemon, specifying the current list of neighbors that the node has. 
The GUI(s) that are listening to that daemon, then draw the neighbors in a way that is relevent for that particular GUI.
\\\\
Usage: 
{\tt showColor -s server [optional args] nbr1 nbr2 nbr3 ... nbrN}
\\\\
Required Arguments: 
\begin{itemize}
\item {\tt -s, --server=address}, The addres of the node running watcherd.
\item {\tt nbr1 nbr2 nbr3 ... nbrN} - the list if neighbors by ipaddress.
\end{itemize}
Optional args:
\begin{itemize}
\item {\tt -l, --layer=layer}, the layer that these neighbors should show up on when displayed in the GUI(s).
\item {\tt -p, --logProps=log.propertiesFile}, the log properties file to use.
\item {\tt -f, --fromNode=fromNodeAddr}, the node that has these neighbors, if not given the local node is assumed.
\end{itemize}
Examples:
\begin{itemize}

\item This tells the GUI(s) that are listening to the daemon running on 'glory' the local test node has neighbors 192.168.1.101 and 192.168.1.102

{\tt sendConnectivityMessage -s glory 192.168.1.101 192.168.1.102}

\item This tells the GUI(s) that are listening to the daemon running on 'glory' the local test node 192.168.1.101 has neighbor nodes 192.168.1.110-192.168.1.115 and they should be displayed on the "children" layer. (Note that 192.168.1.11\{0..5\} is a bashism which expands to the sequenctial list of nodes 192.168.1.110-192.168.1.115.) 

{\tt sendConnectivityMessage -s glory -l children -f 192.168.1.101 192.168.1.11\{0..5\}}

\end{itemize}
