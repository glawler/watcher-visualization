\documentclass{article}

\pdfpagewidth 8.5in
\pdfpageheight 11in

% \setlength\topmargin{0in}
% \setlength\headheight{0in}
% \setlength\headsep{0in}
% \setlength\textheight{7.7in}
\setlength\textwidth{6.5in}
\setlength\oddsidemargin{0in}
\setlength\evensidemargin{0in}
% \setlength\parindent{0.25in}
% \setlength\parskip{0.25in} 

\usepackage{makeidx}
\makeindex
\usepackage[dvips]{graphicx}

\usepackage{url}

\pagestyle{headings}

\author{Geoff Lawler \url{<geoff.lawler@cobham.com>} \and {Michael Elkins \url{<michael.elkins@cobham.com>}}}

\title{Watcher User's Guide}

\begin{document}
\maketitle
\newpage 

\tableofcontents
\newpage

\section{Introduction}

This is the introduction. 

\section{General Watcher Information}
\subsection{Watcher Architecture and Communications}
Give info about the watcher arch here: test node connect to daemon. GUIs connect to daemons and request message streams. Explain what message
streams are. 

\subsection{Obtaining and Building Watcher}
Mention GPL'd-ness of the watcher. Give reference to repo location. 

Give list of dependencies.

Give simple build instructions: ./autogen.sh, ./configure, make, sudo make install, cd clients/legacyWatcher, make, make install \ldots.

\subsection{Log Property Files}
Give general intro with examples of the standard format and workings of the log.properties file. (Just like Java if peeps know that.)

\subsection{Configuration (cfg) Files} 
explain layout of standard config files and arguments that most (all?) watcher modules understand (-f cfg). 

\subsection{Command Line Arguments}
Most binaries take a {\tt -c} (or {\tt -f}) arugment which gives the location of the {\tt cfg} file. Mention this. Also note that most binaries 
just take that option and nothing else. 

\section{Test Node Components}
\subsection{Watcher API}
Mention watcher messages and give types. Mention something else that's interesting and useful. 

\subsection{Scripting Interface}
For each watcher message there is a command line binary to send that message. These binaries can be used directly in shell scripts or invoked via a system call from most scripting languages to send 
an instance of that message to a running watcher daemon instance. Each binary allows the user to specify the content of the message and the daemon instance to send the message to. 

In many cases, the node that the message ``comes from'' can be set as well. This allows a user on any machine that can connect to a watcher daemon, the ability to modify nodes, edges, lables, etc of any 
test node. This is useful for debugging or real time modification of an aspect of the test bed. For instance a single machine could monitor traffic rates between nodes and update 
the edges between those nodes with the current traffic rate. 
\\\\
The available commands are:
\begin{itemize}
\item sendColorMessage (page \pageref{sendColorMessage})
\item sendEdgeMessage (page \pageref{sendEdgeMessage})
\item sendGPSMessage (page \pageref{sendGPSMessage})
\item sendConnectivityMessage (page \pageref{sendConnectivityMessage})
\item sendDataPointMessage (page \pageref{sendDataPointMessage})
\item sendLabelMessage (page \pageref{sendLabelMessage})
\end{itemize}
The following pages give details for each command.
\newpage
\label{sendColorMessage}
\subsubsection{sendColorMessage}
{\bf sendColorMessage} is a test node command line program that sends a ColorMessage message to the watcher daemon, specifing that a node should change it's color. 
\\\\
Usage: 
{\tt sendColorMessage -s server -c color [optional args]}
\\\\
Required Arguments:
\begin{itemize}
\item {\tt -c, --color=color}, The color of the node. Can be ROYGBIV or RGBA format, string or hex value. Supports transparency. 
\item {\tt -s, --server=address}, The addres of the node running watcherd
\end{itemize}
Optional Arguments:
\begin{itemize}
\item {\tt -n, --node=address}, The node to change color, if empty the local node's address is used
\item {\tt -f, --flash=milliseconds}, Flash between the new color and the orginal color every milliseconds seconds, 0 for no flash.
\item {\tt -x, --expiration=seconds}, How long in seconds to change the color. 0==forever
\item {\tt -p, --logProps}, log.properties file, which controls logging for this program
\item {\tt -h, --help}, Show help message
\end{itemize}
Examples:
\begin{itemize}
\item This tells the GUI(s) that are listening to the daemon running node {\em glory} the node at 192.168.1.101 should be drawn in blue:

{\tt showColor -s glory -c blue -n 192.168.1.101}

\item This tells the GUI(s) that are listening to the daemon running node {\em glory} the node at 192.168.1.101 should be drawn in a transparent blue. 
Transparent format is R.G.B.A, where {\em A} is alpha transparency:

{\tt showColor -s glory -c 0.0.255.64 -n 192.168.1.101}
  
\item This tells the GUI(s) that are listening to the daemon running node {\em glory} the node at 192.168.1.107 should be drawn in green for 5 seconds:
 
{\tt showColor -s glory -c green -n 192.168.1.107 --expiration 5000}
 
\item This tells the GUI(s) that are listening to the daemon running node {\em glory} the node at 192.168.1.104 should flash for 10 seconds:
 
{\tt showColor --server glory --color green --node 192.168.1.104 --flash --expiration 10000}

\end{itemize}

\newpage
\label{sendEdgeMessage}
\subsubsection{sendEdgeMessage}

{\bf sendEdgeMessage} is a test node command line program that sends a GPSMessage message to a watcher daemon, specifing a node's current GPS coordinates.
\\\\
Usage: 
{\tt sendEdgeMessage -s server -t tail [optional args]}
\\\\
Required Arguments:
\begin{itemize}
\item {\tt -s, --server=address}, The address or name of the node running watcherd to which the message is sent.
\item {\tt -t, --tail=address}       The node to attach the tail of the edge to. If no head is given, the local node is used.
\end{itemize}
Optional Arguments:
\begin{itemize}
\item {\tt -h, --head=address}       The node to attach the head of the edge to.
\item {\tt -c, --color=color}        The color of the edge. Can be ROYGBIV or RGBA format, string or hex value. Supports transparent colors.
\item {\tt -w, --width=width}        The width of the edge in some arbitrary, unknown unit.
\item {\tt -y, --layer=layer}        Which layer the edge is on in the GUI.
\item {\tt -d, --bidirectional=bool} Is this edge bidirectional or unidirectional. Use 'true' for true, anything else for false.
\item {\tt -l, --label=label}        The text to put in the middle label (This program only supports creating a middle label, although the message supports labels on node1 and node2 as well. May add that later)
\item {\tt -f, --labelfg=color}      The foreground color of the middle label. Can be ROYGBIV or RGBA format, string or hex value. Supports transparent colors.
\item {\tt -b, --labelbg=color}      The background color of the middle label. Can be ROYGBIV or RGBA format, string or hex value. Supports transparent colors.
\item {\tt -z, --fontSize=size}      The font size of the middle label
\item {\tt -x, --expiration=seconds} How long in milliseconds to diplay the edge
\item {\tt -p, --logProps}           log.properties file, which controls logging for this program
\end{itemize}
Examples:
\begin{itemize}
\item Draw an edge between node 101 and node 102 on the "QoS" layer and make the color a translucent red.

{\tt sendEdgeMessage -s glory -h 192.168.1.101 -t 192.168.1.102 -l QoS -c 255.0.0.64}
\end{itemize}


\newpage
\label{sendGPSMessage}
\subsubsection{sendGPSMessage}
{\bf sendGPSMessage} is a test node command line program that sends a GPSMessage message to a watcher daemon, specifing a node's current GPS coordinates.
\\\\
Usage: 
{\tt sendGPSMessage -s server -x value -y value -z value [optional args]}
\\\\
Required Arguments:
\begin{itemize}
\item {\tt -s, --server=address}, The address or name of the node running watcherd to which the message is sent.
\item {\tt -x, --latitude=value}, The latitude of the node.
\item {\tt -y, --longitude=value}, The longitude of the node.
\item {\tt -z, --altitude=value}, The altitude of the node.
\end{itemize}
Optional Arguments:
\begin{itemize}
\item {\tt -n, --fromNode=address|name}, The node that the coordinates refer to. If not given, assume the local node. 
\item {\tt -l, --logProps}, log.properties file, which controls logging for this program
\item {\tt -h, --help}, Show help message
\end{itemize}
Examples:
\begin{itemize}
\item This tells the GUI(s) attached to the watcherd on glory that node 192.168.1.101 is now at 79.23123123, 43.123123123, 20

{\tt sendGPSMessage -s glory -n 192.168.1.101 -x 79.23123123 -y 43.123123123 -z 20}

\item This tells the GUI(s) attached to the watcherd on glory that the local node is at 0.0123123 0.123123123 123

{\tt sendGPSMessage --server glory -n 192.168.1.101 -x 0.0123123 -y 0.123123123 -z 123}

\end{itemize}

\newpage
\label{sendConnectivityMessage}
\subsubsection{sendConnectivityMessage}
{\bf sendConnectivityMessage} is a test node command line program that sends a watcher::event::ConnectivityMessage message to the watcher daemon, specifying the current list of neighbors that the node has. 
The GUI(s) that are listening to that daemon, then draw the neighbors in a way that is relevent for that particular GUI.
\\\\
Usage: 
{\tt showColor -s server [optional args] nbr1 nbr2 nbr3 ... nbrN}
\\\\
Required Arguments: 
\begin{itemize}
\item {\tt -s, --server=address}, The addres of the node running watcherd.
\item {\tt nbr1 nbr2 nbr3 ... nbrN} - the list if neighbors by ipaddress.
\end{itemize}
Optional args:
\begin{itemize}
\item {\tt -l, --layer=layer}, the layer that these neighbors should show up on when displayed in the GUI(s).
\item {\tt -p, --logProps=log.propertiesFile}, the log properties file to use.
\item {\tt -f, --fromNode=fromNodeAddr}, the node that has these neighbors, if not given the local node is assumed.
\end{itemize}
Examples:
\begin{itemize}

\item This tells the GUI(s) that are listening to the daemon running on 'glory' the local test node has neighbors 192.168.1.101 and 192.168.1.102

{\tt sendConnectivityMessage -s glory 192.168.1.101 192.168.1.102}

\item This tells the GUI(s) that are listening to the daemon running on 'glory' the local test node 192.168.1.101 has neighbor nodes 192.168.1.110-192.168.1.115 and they should be displayed on the "children" layer. (Note that 192.168.1.11\{0..5\} is a bashism which expands to the sequenctial list of nodes 192.168.1.110-192.168.1.115.) 

{\tt sendConnectivityMessage -s glory -l children -f 192.168.1.101 192.168.1.11\{0..5\}}

\end{itemize}

\newpage
\label{sendDataPointMessage}
\subsubsection{sendDataPointMessage}
{\bf sendDataPointMessage} is a test node command line program that sends a DataPointMessage message to the watcher daemon, specifing a set of timestamped data point(s) for the node. The data point is labeled with a string saying what the data points represent.  The GUI(s) then display this information is some way. In the case of the legacy watcher [\ref{LegacyWatcher}], 2D scrolling graphs are displayed showing past data points. The legacy watcher assumes that data points are given once a second.
\\\\
Usage: 
{\tt sendDataPointMessage -s server -g name [optional args] -d dp1 -d dp2 ... -d dpN}
\\\\
Required Arguments:
\begin{itemize}
\item {\tt -s address|name}, The address or name of the node running watcherd.
\item {\tt -g name}, the "graphname" of the data - what the data is measuring.
\item {\tt -d datapoint}, a single data point measuring something.
\end{itemize}
Optional args:
\begin{itemize}
\item {\tt -n, --node=address}, the node the data is from 
\item {\tt -h, --help}, Show help message
\end{itemize}
Examples:
\begin{itemize}
\item Update the watcher about the current CPU usage on the machine 192.168.1.105.  
    
{\tt sendDataPointMessage -s glory -g "CPUUsage" -n 192.168.1.105 -d .45432}

\item Update the watcher about the current number of user's logged in to the local machine.   

{\tt sendDataPointMessage -s glory -g "LoggedInUsers" -d 23}

\item Update the watcher about the local node's current level of self satisifation.

{\tt sendDataPointMessage -s glory -g "BoyAmIGreatLevel" -d 23}

\end{itemize}

\newpage
\label{sendLabelMessage}
\subsubsection{sendLabelMessage}
{\bf sendLabelMessage} is a test node command line program that sends a watcher::event::LabelMessage message to the watcher daemon, specifing that a label should be attached to the specified node (or float if given coords).

If address is specified, the label will attach to the node with that address. If cooridinates are
specified, the label will float at those coordinates. The node address takes precedence. If neither
option is specified, the label will attach to the node from which the message saw sent.
\\\\
Usage: 
{\tt sendLabelMessage -s server -l label [optional args]}
\\\\
Required Arguments:
\begin{itemize}
\item {\tt -l, --label=text}, The text of the label
\item {\tt -s, --server=address}, The address|name of the node running watcherd, the server.
\end{itemize}
Optional args:
\begin{itemize}
\item {\tt -n, --node=address}, The node to change color, if empty the local node's address is used
\item {\tt -x, --latitude=coord}        The latitude to float the node at.
\item {\tt -y, --longitude=coord}       The longitude to float the node at.
\item {\tt -z, --altitiude=coord}       The altitude to float the node at.
\item {\tt -t, --fontSize=size}         The font size of the label
\item {\tt -f, --foreground=color}      The foreground color of the label. Can be ROYGBIV or RGBA format, string or hex value.
\item {\tt -b, --background=color}      The background color of the label. Can be ROYGBIV or RGBA format, string or hex value.
\item {\tt -e, --expiration=seconds}    How long in millisecond to diplay the label
\item {\tt -r, --remove}                Remove the label if it is attached
\item {\tt -L, --layer=layer}           Which layer the label is part of. Default is "Physcial".
\item {\tt -x, --expiration=seconds}, How long in seconds to change the color. 0==forever
\item {\tt -p, --logProps}, log.properties file, which controls logging for this program
\item {\tt -h, --help}, Show help message
\end{itemize}
Examples:
\begin{itemize}
\item sendLabelMessage -s glory -n 192.168.1.102 -l "Correlation Layer" -e 1500 -f red -b green -L Correlation
\item sendLabelMessage -s glory -n 192.168.1.102 -l "Physical Layer" -e 1500 -L Physical
\item sendLabelMessage -s glory -n 192.168.1.104 -l "Attack Detected" -f yellow -b blue -L Physical 
\end{itemize}



\section{The Watcher Daemon}

The Watcher Daemon is responsible for collecting events from the Test Nodes and
sending event streams to the Watcher GUIs.  Events from the Test Nodes are
stored in an SQLite databased (named "event.db" by default).  The Watcher Daemon
determines whether a connection is a Test Node or GUI by the type of the first
event received.

When recording events from the Test Nodes into the database, events are
appended to the existing database, or a new database is created if it does not
exist.  The Watcher Daemon may also be invoked in read-only database mode using
the command line option -r or --read-only, in which case events are not stored
in the database.  Read-only mode is useful particularly when replaying events
from a database from some time in the past.  In this case, it may not make
sense to append any current event stream from the Test Nodes when a large time
gap exists between past and present runs.

\subsection{Live Mode}

When a GUI connects to the Watcher Daemon, it will by default subscribe to the
live event stream coming from the Test Nodes.  In this case, the Watcher Daemon
is simply retransmitting received events to all listening GUIs rather than
fetching events from the database.  The events are also stored in the database
for later replay.

If a GUI pauses, rewinds, or slows playback, it will switch to Playback mode.

\subsection{Playback Mode}

In Playback Mode, the Watcher Daemon fetches events from the database and sends
them to the GUI.  Each GUI connection has an independent notion of the current
playback time offset, direction and speed.  Stopping playback in one GUI will
not cause playback to stop in another GUI.

The Watcher Daemon will automatically pause Playback when the last event from
the database has been sent to the GUI.  Thus, if a GUI were playing at a time
offset near the end of the database, and faster than real time, the GUI will be
paused when the last event is sent, even if additional events arrive from Test
Nodes.

\section{Watcher GUIs}

Talk about GUIs here. Legacy watcher is OpenGL, others are engine based and just barely started proof of concept.

\subsection{Legacy Watcher}
\label{LegacyWatcher}
\begin{figure}
\centering
\includegraphics[width=0.8\textwidth]{legWatcherGUI.eps}
\caption{The legacy watcher showing a running instance of showClock.}
\end{figure}

\subsection{ogreWatcher}
\begin{figure}
\centering
\includegraphics[width=0.8\textwidth]{ogreWatcherGUI.eps}
\caption{The ogreWatcher showing a running instance of showClock.}
\end{figure}

Built with OGRE.
\subsection{Watcher3d}
Built with Delta-3d.

\printindex

\end{document}
