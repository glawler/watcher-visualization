\subsection{messageStream2Text}

{\it messageStream2Text} is not your typical GUI in that it has no graphical interface. It's in the GUI section though as it uses 
the GUI watcher API to talk to a running watcher daemon. It was written (and is used) as a tool to debug the message stream interface. It simply
connects to a daemon instance, requests a message stream, and dumps all the recieved messages to the console on stdout. Think of it as a 
tcpdump for watcher message streams. 

{\it messageStream2Text} source code can be found in {\tt .\slash src\slash clients\slash messageStream2Text}. 

\subsubsection{Configuration}

\begin{itemize}
\item {\tt -h} or {\tt --help}, show a usage message and exit. 
\item {\tt -c configfile}, gives the location of the configuration file. If not given a default one will be created, used, and saved on program exit.
\end{itemize}

\subsubsection{Command Line Options}

\begin{itemize}
\item {\tt server}, name or ipaddress of the server to connect to.
\item {\tt service}, name of service (usaully "watcherd") or port number on which the server is listening.
\end{itemize}

