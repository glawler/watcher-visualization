\subsubsection{watcherHierarchyClient}
\label{watcherHierarchyClient}

The {\it watcherHierarchyClient} daemon is used to support a test bed running the the dynamic hierarchy software. 
The watcher was orgininally written with the hieracht daemons as a transport layer and as part of a 
dynamic hierarchy MANET platform. Thus the {\it watcherHierarchyClient} daemon is used for backwards compatibility. 
The basic idea of the {\tt watcherHierarchyClient} is that it connects to a hierarchy instance, subscribes
to all the messages that a watcher GUI may care about and when it receives such a message it translates it 
into something that the (new) watcher system can understand. 

{\it watcherHierarchyClient} is the glue between hierachy land and watcher land. {\it watcherHierarchyClient}, when started, connects
to a running hierarchy daemon and subscribes to all watcher related messages. When it receives a watcher related
messages, it converts the message into something the watcher system can understand and sends it to the watcher daemon
that it is connected to. {\it watcherHierarchyClient} is meant to offer backward compatibility to all "old style" watcher 
clients. It acts as a go-between between old hierarchy messages and the new watcher messages.
\\\\
Usage:
{\tt watcherHierarchyClient -s watcher\_daemon\_name\_or\_address -u hierachy\_daemon\_node\_address}
\\\\
Arguments:
\begin{itemize}
\item {\tt -s}, The address or name of the node running the watcher daemon, watcherd.
\item {\tt -u}, The address of the node running the hierachy daemon.
\end{itemize}

