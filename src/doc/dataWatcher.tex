\subsection{Data Watcher}
\label{DataWatcher}

The Data Watcher (aka dataWatcher), displays periodically generated data points from the test nodes. It gives a global and local window
into the data point stream. The data points are opaque, the meaning attached to the data is given by the test node in the name of the data point stream.
The Data Watcher lets the user choose specific test nodes' data to view. The local window can be moved to show an arbitrary slice of the data point
stream.

{\it dataWatcher} source code can be found in {\tt .\slash src\slash
clients\slash dataWatcher}. The binary produced after building is 
named {\tt dataWatcher}. This is a Qt-based application, and thus uses the Qt build system, qmake. To build: {\tt qmake, make}.

\subsubsection{Command Line Options}
\begin{itemize}
\item {\tt -h, --help}, show a usage message and exit. 
\item {\tt -s, --server}, connect to the watcher daemon at this address or hostname. Supports host:port, address:port syntax.
\item more ...
\end{itemize}

\subsubsection{Configuration}
\begin{itemize}
\item {\tt server}, name or ipaddress of the server to connect to.
\end{itemize}

\subsubsection{Usage}

Push the buttons, slide the slider, etc.



