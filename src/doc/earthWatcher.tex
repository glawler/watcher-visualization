\subsection{Earth Watcher}
\label{EarthWatcher}

The Earth Watcher is a daemon application for Linux systems that allows
visualization of Watcher events in the Google Earth GUI.

{\it earthWatcher} source code can be found in {\tt .\slash src\slash
clients\slash earthWatcher}. The binary produced after building is 
named {\tt earthWatcher}. 

\subsubsection{Command Line Options}
\begin{itemize}
\item {\tt -h} or {\tt --help}, show a usage message and exit. 
\item {\tt -c} of {--configFile}, gives the location of the configuration file. If not given a default one will be created, used, and saved on program exit.

\item {\tt -a} or {\tt --latoff}, translate GPS coordinates relative to a given latitude
\item {\tt -A} or {\tt --altoff}, translate GPS coordinates relative to the given altitude
\item {\tt -c} or {\tt --config}, specify a configuration file (default:
earthWatcher.cfg)
\item {\tt -d} or {\tt --speed}, specify the event playback rate
\item {\tt -h} or {\tt --help}, display this help message
\item {\tt -i} or {\tt --icon-scale}, adjust the size of node icons
\item {\tt -I} or {\tt --icon-path}, specify the node icon to use
\item {\tt -o} or {\tt --output}, specifies the output KML file
\item {\tt -O} or {\tt --lonoff}, translate GPS coordinates relative to a given longitude
\item {\tt -r} or {\tt --refresh}, write the the output every SECS seconds
\item {\tt -s} or {\tt --server}, connect to the watcher server on the given host
\item {\tt -S} or {\tt --seek}, start event playback at the specified timestamp.  May be specified relative to the first and last timestamps in the Watcher database by prefixing the offset with "+" or "-". Example: +5000 means 5 seconds after the first event in the database.
\item{\tt  -t} or {\tt --steps}, number of points to use when drawing a spline (default: 2)

\end{itemize}

\subsubsection{Configuration}
\begin{itemize}
\item {\tt server}, name or ipaddress of the server to connect to.
\item {\tt service}, name of service (usaully "watcherd") or port number on which the server is listening.
\item {\tt layers}, a listing of layers from the Layers menu. All layers the watcher knows about will show up here in a {\tt layername = bool} pair. 
If the bool is {\tt true}, the layer will be active (and shown), if {\tt false} the layer will not be shown. Every {\tt layername} will become
a folder in the {\tt Places} menu in the Google Earth GUI.
\item {\tt layerPadding = float}, how much padding (in feet) to place between layers. 
\item {\tt latOff = float}, translate GPS coordinates relative to a given latitude.
\item {\tt lonOff = float}, translate GPS coordinates relative to a given longitude.
\item {\tt outputFile = string}, specifies the output KML file.
\item {\tt iconPath = string}, specify an alternate icon to use for nodes.
\item {\tt splineSteps}, the number of points in a spline when drawing edges.
\end{itemize}

\subsection{Reloading Configuration File}

EarthWatcher checks for changes in the configuration file while running.  The purpose of this feature is to allow for
defined layers to be toggled on or off during runtime.  No options other than the layers are checked for changes during runtime.

\subsection{Translating GPS coordinates}

If the GPS coordinates stored in the Watcher database during a test run are not
accurate, the EarthWatcher has options to translate the coordinates relative to
some location on the Earth.  For example, the origin is located in the Atlantic
Ocean off the coast of West Africa.  Using the {\tt latOff} and {\tt lonOff}
options can place the nodes in a more geographically interesting location.
